\section{Exercício 1.5}
\paragraph{Escrever novos exemplos de duas sentenças de S que são fórmulas e de duas sentenças que não são fórmulas. Cada exemplo deve ser justificado de forma circunstanciada.}

\begin{quote}
	Sintaxe de 	\textfrak{S}:
	\begin{enumerate}[label=\roman*., itemsep=0pt, topsep=0pt]
		\item Se \textit{u} e \textit{v} são termos de	\textfrak{S}, então \textit{u} = \textit{v} e \textit{u} $\in$ \textit{v} são \textit{fórmulas atômicas} de \textfrak{S}.
		\item Toda fórmula atômica de \textfrak{S} é \textit{fórmula} de \textfrak{S}.
		\item Se $\mathcal{A}$ e $\mathcal{B}$ são fórmulas de \textfrak{S} e \textit{u} é uma variável, então as sentenças $\neg$($\mathcal{A}$), ($\mathcal{A}$ $\land$ $\mathcal{B}$), ($\mathcal{A}$ $\lor$ $\mathcal{B}$), ($\mathcal{A}$ $\Rightarrow$ $\mathcal{B}$), ($\mathcal{A}$ $\Leftrightarrow$ $\mathcal{B}$), $\forall$\textit{u}($\mathcal{A}$), $\exists$\textit{u}($\mathcal{A}$) são fórmulas de \textfrak{S}.
		\item Apenas as sentenças de \textfrak{S} que seguem os itens acima são fórmula de \textfrak{S}.
	\end{enumerate}
\end{quote}

\subsection{primeira fórmula}
{$\forall$\textit{x}(\textit{x} = $\omega$)}
\newline
\textit{x} é uma abreviação de variável e $\omega$ é uma constante; logo, \textit{x} e $\omega$ são termos. O item I da sintaxe \textfrak{S} garante que \textit{x} = $\omega$ é uma fórmula atômica. O item II enuncia que toda fórmula atômica é fórmula de \textfrak{S}. A sentença $\forall$\textit{u}($\mathcal{A}$) é fórmula de \textfrak{S}(item III), então {$\forall$\textit{x}(\textit{x} = $\omega$)} também é fórmula de \textfrak{S}. O item IV conclui que as sentenças de \textfrak{S} que seguem os itens I, II, III são fórmulas de \textfrak{S}; logo, \newline{$\forall$\textit{x}(\textit{x} = $\omega$)} é fórmula de \textfrak{S}.

\subsection{segunda fórmula}
{$\exists$\textit{$x_1$}(\textit{$x_1$} $\in$ \textit{$x_2$} $\land$ $\lnot$($x_2$ = \textit{$x_3$}}))
\newline
\textit{$x_1$}, \textit{$x_2$} e \textit{$x_3$} são váriaveis; logo, são termos. O item I da sintaxe \textfrak{S} garante que \textit{$x_1$} $\in$ \textit{$x_2$} e $x_2$ = \textit{$x_3$} são fórmulas atômicas(item II), portanto são fórmulas de \textfrak{S}. No item III temos:  $\neg$($\mathcal{A}$), ($\mathcal{A}$ $\land$ $\mathcal{B}$) e  $\forall$\textit{u}($\mathcal{A}$); então $\lnot$($x_2$ = \textit{$x_3$}), \newline(\textit{$x_1$} $\in$ \textit{$x_2$} $\land$ $\lnot$($x_2$ = \textit{$x_3$}) e {$\exists$\textit{$x_1$}(\textit{$x_1$} $\in$ \textit{$x_2$} $\land$ $\lnot$($x_2$ = \textit{$x_3$}})) são fórmulas de \textfrak{S}. O item IV conclui que as sentenças de \textfrak{S} que seguem os itens I, II, III são fórmulas de \textfrak{S}; logo,  {$\exists$\textit{$x_1$}(\textit{$x_1$} $\in$ \textit{$x_2$} $\land$ $\lnot$($x_2$ = \textit{$x_3$}})) é fórmula de \textfrak{S}.

\subsection{primeira sentença que não é fórmula}
{$\exists$$\forall$\textit{x}(\textit{x} = \textit{y})}
\newline
Apesar de \textit{x} = \textit{y} ser uma fórmula atômica, a sentença {$\exists$$\forall$\textit{x}(\textit{x} = \textit{y})} não é fórmula, pois o item III da definição requer que o quantificador $\exists$ tenha uma variável imediatamente à direita.

\subsection{segunda sentença que não é fórmula}
{$\exists$\textit{x} $\lor$ $\exists$\textit{y}$\forall$\textit{x}(\textit{x} $\in$ \textit{y})}
\newline
O item III da definição de fórmula exige que $\exists$\textit{x} seja seguido de uma sentença $\mathcal{A}$, assim como em $\exists$\textit{y}$\forall$\textit{x}(\textit{x} $\in$ \textit{y}), em que $\mathcal{A}$ é fórmula - descrito no item III como $\exists$\textit{u}($\mathcal{A}$). No caso da sentença $\lor$ $\exists$\textit{y}$\forall$\textit{x}(\textit{x} $\in$ \textit{y}), o conectivo lógico $\lor$(disjunção) é o elemento que impede a classificação da sentença como fórmula.