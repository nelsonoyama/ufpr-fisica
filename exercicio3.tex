
\paragraph{Para \textit{x} = \{3,2,5\} e \textit{y} = \{3,7\}, dar exemplo de relação \textit{r} : \textit{x} $\rightarrow$ \textit{y} que não seja função e relação  \textit{s} : \textit{x} $\rightarrow$ \textit{y} que seja função.}

\begin{quote}
	
	O produto cartesiano não é uma operação comutativa, por isso permite distinção entre domínio e co-domínio de uma relação \textit{r}.
	%Seja \textit{r}  : \textit{x} $\rightarrow$ \textit{y} uma relação. Dizemos que \textit{r} e uma função \textit{sss} para todo \textit{a} pertencente a \textit{x} existe um e apenas um \textit{b} pertencente a \textit{y} tal que (\textit{a},\textit{b}) $\in$ \textit{r}.
\end{quote}


\subsection{Exemplo de relação \textit{r} : \textit{x} $\rightarrow$ \textit{y} que não é função}
	Uma relação \textit{r} com domínio \textit{x} e co-domínio \textit{y} é um subconjunto de pares ordenados do produto cartesiano \textit{x} \texttimes \,\textit{y}. Qualquer subconjunto desse produto cartesiano será uma relação \textit{r} : \textit{x} $\rightarrow$ \textit{y}.\newline
	Dizemos que \textit{r} e uma função \textit{sss} para todo \textit{a} pertencente a \textit{x} existe um e apenas um \textit{b} pertencente a \textit{y} tal que (\textit{a},\textit{b}) $\in$ \textit{r}. Temos \textit{x} \texttimes \,\textit{y} = \{(3,3), (3,7), (2,3), (2,7), (5,3), (5,7)\}).
	\paragraph{\{(2,3), (2,7)\}} é uma relação \textit{r} com domínio em \textit{x} e co-domínio em \textit{y}, pois é um subconjunto de \textit{x} \texttimes \,\textit{y}. Não é uma função \textit{r} com domínio em \textit{x} e co-domínio em \textit{y}, pois para o domínio \textit{x}, nem todo \textit{a}(elementos do domínio \textit{x}) tem uma imagem no co-domínio \textit{y}. 
	%Além disso, para o domínio \textit{x} há mais de um elemento no co-domínio \textit{y}, invalidando a definição de função.
	 

\subsection{Exemplo de relação  \textit{s} : \textit{x} $\rightarrow$ \textit{y} que é função.}
	\paragraph{\{(3,3), (2,3), (5,7)\}} é uma função \textit{s} com domínio em \textit{x} e co-domínio em \textit{y}, pois é um subconjunto de \textit{x} \texttimes \,\textit{y}. Além disso, na função \textit{s} com domínio em \textit{x} e co-domínio em \textit{y}, todos os elementos \textit{a} do domínio \textit{x} tem uma, e apenas uma, imagem \textit{b }no co-domínio \textit{y}.

%	é um exemplo de função \textit{s} de \textit{x} em \textit{y}. Os elementos do domínio são 3(par ordenado (3,3)) e 5(par ordenado (5,7)). Para o primeiro par ordenado nós temos uma imagem 3 e para o segundo par ordenado temos uma imagem 7. Nesse caso, ambos pares ordenados pertencem a relação \textit{s} e para todo \textit{a}, existe um e apenas um \textit{b} pertencente a \textit{y}, satisfazendo a definição de função \textit{s} de \textit{x} em \textit{y}. 
	

