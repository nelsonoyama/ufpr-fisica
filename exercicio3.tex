
\paragraph{Para \textit{x} = \{3,2,5\} e \textit{y} = \{3,7\}, dar exemplo de relação \textit{r} : \textit{x} $\rightarrow$ \textit{y} que não seja função e relação  \textit{s} : \textit{x} $\rightarrow$ \textit{y} que seja função.}

\begin{quote}
	
	Seja \textit{r}  : \textit{x} $\rightarrow$ \textit{y} uma relação. Dizemos que \textit{r} e uma função \textit{sss} para todo \textit{a} pertencente a \textit{x} existe um e apenas um \textit{b} pertencente a \textit{y} tal que (\textit{a},\textit{b}) $\in$ \textit{r}.
\end{quote}


\subsection{Exemplo de relação \textit{r} : \textit{x} $\rightarrow$ \textit{y} que não é função}
	\paragraph{\{(3,3), (2,3)\}}
	é um exemplo de relação \textit{r} com domínio \textit{x} e co-domínio \textit{y}. Na definição de função, dizemos que \textit{r} é uma função \textit{sss} para todo \textit{a} pertencente a \textit{x} existe um e apenas um \textit{b} pertencente a \textit{y} tal que (\textit{a},\textit{b}) $\in$ \textit{r}. Nesse caso, existem dois inteiros, representados pelo 3 inteiro, em \textit{b}, invalidando a classificação do conjunto como função.
	

\subsection{Exemplo de relação  \textit{s} : \textit{x} $\rightarrow$ \textit{y} que é função.}
	\paragraph{\{(3,3), (5,7)\}}
	é um exemplo de relação \textit{s} com domínio \textit{x} e co-domínio \textit{y}. Na definição de função, dizemos que \textit{s} é uma função \textit{sss} para todo \textit{a} pertencente a \textit{x} existe um e apenas um \textit{b} pertencente a \textit{y} tal que (\textit{a},\textit{b}) $\in$ \textit{s}. Nesse caso, as condições para classificar como função são válidas. 

