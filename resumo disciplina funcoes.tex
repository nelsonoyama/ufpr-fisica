\documentclass[a4paper]{article}
\usepackage[brazilian]{babel}
\usepackage[utf8]{inputenc}
\usepackage{enumitem}
\usepackage{yfonts}
\usepackage[T1]{fontenc}
\usepackage{color}
\usepackage{geometry}
\geometry{
	a4paper,
	total={210mm,297mm},
	left=30mm,
	top=30mm,
	right=20mm,
	bottom=20mm,
}

\author{Nelson Oyama}
\title{Resumo livro Matemática Pandêmica + aula}
\date{out/2021}
\begin{document}
	\maketitle
	\paragraph{}O objetivo principal do livro é fornecer ferramentas básicas para a autonomia do aluno, no estudo da matemática.
	Nessa leitura o professor critica a doutrinação, afirmando que o ensino da matemática não é um procedimento para aprovação em exames. Durante a aula, foi citado Richard Feymann, prêmio nobel de física. Durante uma breve estada no Brasil, Feymann ficara intrigado com os alunos brasileiros em relação ao aprendizado, a mesma pergunta feita de maneiras diferentes não era entendida pelos alunos brasileiros, ainda que respondessem a primeira pergunta.
	\paragraph{}Há destaque para a extensa utilização da matemática: física, química, computação, etc... Também é preciso destacar um benefício pessoal do aprendizado: estabelecer uma leitura senstata e bem informado sobre o mundo. Segundo o autor: senso comum não é um bom ponto de partida para uma visão racional do mundo.
	\paragraph{}A Linguagem e lógica matemática serão trabalhadas nesse estudo, pois a língua portuguesa está comprometida com a semântica. É necessário uma linguagem que não tenha contraparte semântica. Isso permite interpretar uma fórmula de várias maneiras. Além disso, uma linguagem formal não abre espaço para ambiguidade, ainda que posso acontecer de maneira sutil. A lógica permite a utilização de regras de inferência, as quais viabilizam relações entre fórmulas.
	\paragraph{}A caminhada de aprendizagem requer:
	\begin{enumerate}
		\item LER: Necessário senso crítico;
		\item PENSAR: "mente aberta", saber lidar com as incertezas;
		\item DIALOGAR: trocar ideias com pessoas que compartilham dos mesmos interesses. O professor cita no livro: "Ciência é um fenômeno social sinérgico".
	\end{enumerate}
	
	\paragraph{}A ideia central é um estudo introdutório bem fundamentado, usando a metodologia da Teoria da História(Robert McKee): um arquitrama composto de três atos. 
	\begin{enumerate}
		\item O primeiro é a apresentação da personagem. Para o estudo, será composto das partes 2 e 3.
		\item No segundo ato, há uma pressão sobre a personagem, que revela o caráter dela diante dessa pressão. Será composto da parte 4.
		\item Finalmente, o terceiro ato é a solução do problema. Nesse caso o autor escreve que isso ainda está em andamento na aventura humana.
	\end{enumerate}
	\paragraph{\textcolor{blue}{Meus comentários: A ideias de diálogo, senso crítico, leitura estão alinhadas com a aula 4 e 5 de História da física. Escola grega buscava explicações naturais para os fenômenos da natureza. Em Mileto: Tales, Anaximandro, Anaximeres conseguiam debater ideias diferentes. Pelo que eu entendi, conheciam bem as obras de seus pares. Debatiam e discutiam o árche, substância primordial. Para Tales era a água, Anaximandro considerava o apeyron e Anaximeres o ar. No livro Matemática pandêmica: doutrinação $\neq$ debate.}}
	
	\newpage
	\begin{quote}
	De acordo com Galileu, tradução livre(material aula Prof Sergio Sanchez - aula online em 05/10/2021)\newline
	"A filosofia[i.e., a física] encontra-se escrita neste grande livro que continuamente se abre perante nossos olhos(isto é, o universo), que não se pode compreender antes de entender a língua e conhecer os caracteres com os quais está escrito. Ele está escrito em língua matemática, os caracteres são triângulos, circunferências e outras figuras geométricas, sem cujos meios é impossível entender humanamente as palavras; sem eles nós vagamos perdidos dentro de um obscuro labirinto."
	\newline
	O Ensaiador(1623).
		\end{quote}
\end{document}

