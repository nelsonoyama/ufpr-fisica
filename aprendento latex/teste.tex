\documentclass[a4paper]{article}
\usepackage[brazilian]{babel}
\usepackage[utf8]{inputenc}
\usepackage{enumitem}
\usepackage{yfonts}
\usepackage[T1]{fontenc}
\newtheorem{teorema}{Teorema}[section]
\newtheorem{definicao}[teorema]{Definição}
\newtheorem{proposicao}{Proposição}[section]
\author{Nelson Oyama}
\title{Dúvidas}
%\title{Bloco 01 da disciplina de CMM012-Funções - Professor Adonai}
%\date{05 de Novembro de 2021}
\date{27/09/2021}

\begin{document}
	\maketitle
	\section{Primeira parte}
	\begin{teorema}[Teorema principal]
		todo quadrado tem quatro lados.
	\end{teorema}
	\begin{definicao}
		Um quadrado é um polígono de quatro lados.
		conteúdo...
	\end{definicao}
		\begin{teorema}
		todo triângula tem três lados.
	\end{teorema}
	\begin{proposicao}
	Um pentágono tem cinco lados.
	\end{proposicao}
	\section{Segunda parte}
			\begin{teorema}
			todo quadrado tem quatro ângulos retos.
		\end{teorema}
		\begin{teorema}
			todo triângula tem três ângulos agudos.
		\end{teorema}
\end{document} 