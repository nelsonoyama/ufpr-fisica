
\paragraph{Fazer a demonstração do Teorema 19. Obviamente essa tarefa exige o emprego dos Teoremas 16, 17 e 18.}

\begin{quote}
	Teorema 16. Seja \textit{f} : $\mathbb{R}$ $\rightarrow$ $\mathbb{R}$ tal que \textit{f(x) = c}. Logo $\lim_{x\rightarrow a}$ \textit{f(x) = c}.
	\newline
	Teorema 17. Seja \textit{f} : $\mathbb{R}$ $\rightarrow$ $\mathbb{R}$ tal que \textit{f(x) = x}. Logo $\lim_{x\rightarrow a}$ \textit{f(x) = a}.
	\newline
	Teorema 18. Se $\lim_{x\rightarrow a}$ \textit{f(x) = \textit{L}} e $\lim_{x\rightarrow a}$ \textit{g(x) = \textit{M}} então:\newline
	\textit{$(i)$} $\lim_{x\rightarrow a}$ \textit{(f + g)(x) = \textit{L}} + \textit{M}, \textit{$(ii)$} $\lim_{x\rightarrow a}$ \textit{(f - g)(x) = \textit{L}} - \textit{M},
	\newline
	\textit{$(iii)$} $\lim_{x\rightarrow a}$ \textit{(fg)(x) = \textit{L}}\textit{M}, 	\textit{$(iv)$} $\lim_{x\rightarrow a}$ \textit{(f/g)(x) = \textit{L}}/\textit{M} (se \textit{M} $\neq$ 0).
	\newline
	Em particular, $\lim_{x\rightarrow a}$ \textit{cf(x) = c\textit{L}}.
\end{quote}

\subsection{Demonstração do Teorema 19}
\textit{Teorema 19. Seja p : $\mathbb{R}$ $\rightarrow$ $\mathbb{R}$ tal que p = $a_{n}$$x^{n}$ + $a_{n-1}$$x^{n-1}$ + $a_{n-2}$$x^{n-2}$ + ... + $a_{3}$$x^{3}$ + $a_{2}$$x^{2}$ + $a_{1}$$x^{1}$ + $a_{0}$, onde $a_{0}$, $a_{1}$, ... ,$a_{n}$ são números reais e n é uma cópia de um número natural entre os reais. Então, $\lim_{x\rightarrow a}$ p(x) = p(a).}
\newline \newline
Aplicando o teorema 18 temos uma soma de monômios, o limite da soma é a soma dos limites: \newline
$\lim_{x\rightarrow a}$ \textit{p(x)} = $\lim_{x\rightarrow a}$$a_{n}$$x^{n}$ + $\lim_{x\rightarrow a}$$a_{n-1}$$x^{n-1}$ + $\lim_{x\rightarrow a}$$a_{n-2}$$x^{n-2}$ + ... + $\lim_{x\rightarrow a}$$a_{3}$$x^{3}$ + $\lim_{x\rightarrow a}$$a_{2}$$x^{2}$ + $\lim_{x\rightarrow a}$$a_{1}$$x^{1}$ + $\lim_{x\rightarrow a}$$a_{0}$
\newline \newline
Novamente com o Teorema 18, temos o limite do produto é o produto dos limites:  \newline
$\lim_{x\rightarrow a}$ \textit{p(x)} = $\lim_{x\rightarrow a}$$a_{n}$.$\lim_{x\rightarrow a}$$x^{n}$ + $\lim_{x\rightarrow a}$$a_{n-1}$.$\lim_{x\rightarrow a}$$x^{n-1}$ + $\lim_{x\rightarrow a}$$a_{n-2}$.$\lim_{x\rightarrow a}$$x^{n-2}$ + ... + $\lim_{x\rightarrow a}$$a_{3}$.$\lim_{x\rightarrow a}$$x^{3}$ + $\lim_{x\rightarrow a}$$a_{2}$.$\lim_{x\rightarrow a}$$x^{2}$ + $\lim_{x\rightarrow a}$$a_{1}$.$\lim_{x\rightarrow a}$$x^{1}$ + $\lim_{x\rightarrow a}$$a_{0}$ 
\newline \newline
No teorema 16 temos o limite de uma constante é igual a constante, $\lim_{x\rightarrow a}$ \textit{c = c}, então: 
\newline
$\lim_{x\rightarrow a}$ \textit{p(x)} = $a_{n}$.$\lim_{x\rightarrow a}$$x^{n}$ + $a_{n-1}$.$\lim_{x\rightarrow a}$$x^{n-1}$ + $a_{n-2}$.$\lim_{x\rightarrow a}$$x^{n-2}$ + ... + $a_{3}$.$\lim_{x\rightarrow a}$$x^{3}$ + $a_{2}$.$\lim_{x\rightarrow a}$$x^{2}$ + $a_{1}$.$\lim_{x\rightarrow a}$$x^{1}$ +$a_{0}$ 
\newline \newline
No teorema 17 temos $\lim_{x\rightarrow a}$ \textit{x = a}. A expressão $x^{n}$ pode ser escrita como $x$$x$$x$...\textit{n} vezes. Usando o Teorema 17 (limite da função identidade) e o Teorema 18 (produto dos limites), temos :\newline
$\lim_{x\rightarrow a}$ \textit{p(x)} = $a_{n}$.$a^{n}$ + $a_{n-1}$.$a^{n-1}$ + $a_{n-2}$.$a^{n-2}$ + ... + $a_{3}$.$a^{3}$ + $a_{2}$.$x^{2}$ + $a_{1}$.$a^{1}$ +$a_{0}$ = \textbf{\textit{p(a)}}


%$\lim\limits_{x\rightarrow a}$\textit{p(x)} = $\lim_{x\rightarrow a}$ \textit{p(x)}


