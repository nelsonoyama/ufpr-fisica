Sabendo que a meia-vida de $^{14}$C é de 5730 anos, qual é a massa final de dois gramas deste isótopo após oitenta milhões de anos? Para resolver este problema empregue o modelo usual de decaimento radioativo, o qual assume que a taxa de variação de massa em relação à passagem de tempo é proporcional à massa.

\subsection{Massa final de 2 gramas após 80 milhões de anos}
Temos que \textit{m(t) = c exp(kt)}. Em t = 0, m(t) = c, então \textit{m(t) = m$_{0}$ exp(kt)} \newline
Calculando a \textit{k}: $\frac{m_0}{2}$ = $m_0$ exp(k 5730)
0,5 = exp(k 5730)\newline
ln(0,5) = ln(exp(k 5730))\newline
-0,693147 = k 5730 \newline
k = -0,000121 anos $^{-1}$ \newline
Calculando a massa final de 2 gramas após 80 milhões de anos: \newline
m(t) = 2 exp(-0,000121 . 80 . 10$^{6}$) \newline
m(t) = 2 exp(-9677,447547) \newline
Usando o teorema de que a exponencial de uma soma é o produto de exponenciais, temos:\newline
m(t) = 2 (exp(-96,774475))$^{100}$ = 2 (9,36223 . 10$^{-4300}$) = 18,7245 . 10$^{-4300}$ g.\newline
Uma vez que a massa de um âtomo esta na ordem de grandeza de 10$^{-24}$g, valores menores do que isso não apresentam significado físico. Se considerarmos um tempo menor, como 80 mil anos, trabalharemos na ordem de 10$^{-4}$ g, resultado que pode ser plausível, mas pode ultrapassar os limites práticos para uma análise confiável.
