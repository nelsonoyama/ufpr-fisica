
\paragraph{Usar o axioma do Par para definir outro exemplo de conjunto.}
\begin{quote}
	Axioma par: $\forall$\textit{x}$\forall$\textit{y}$\exists$\textit{z}$\forall$\textit{t}(\textit{t} = \textit{x} $\lor$ \textit{t} = \textit{y}))
	\newline
	Usando o emprego de definições explícitas abreviativas na linguagem \textfrak{S}, temos:
	\newline
	\textit{z} =\{{\textit{x},\textit{y}}\}  \vdots \, $\forall$\textit{t}(\textit{t} = \textit{x} $\lor$ \textit{t} = \textit{y}))
	\newline
	Os axiomas Extensionalidade e Vazio, respectivamente $\forall$\textit{x}$\forall$\textit{y}$\forall$\textit{z}((\textit{z} $\in$ \textit{x} $\Leftrightarrow$ \textit{z} $\in$ \textit{y}) $\Rightarrow$  \textit{x} = \textit{y}) e $\exists$\textit{x}$\forall$\textit{z}(\textit{z} $\notin$ \textit{x}) permitem a demonstração do Teorema: O conjunto \textit{x} do axioma Vazio é único. Utilizando esse axioma podemos escrever \textit{x} = $\emptyset$ e \textit{y} = $\emptyset$, logo \textit{z} = \{$\emptyset$\}.
	\newline
	Se \textit{z} = \{$\emptyset$\}, podemos criar outro exemplo: \textit{x} = $\emptyset$ e \textit{y} = \{$\emptyset$\}, logo \textit{z} = \{$\emptyset$, \{$\emptyset$\}\}. Pelo axioma da Extensionalidade: \{$\emptyset$, \{$\emptyset$\}\} = \{\{$\emptyset$\}$, \emptyset$\}.

	
\end{quote}
\subsection{outro exemplo de conjunto}

Se  \textit{x} = \{$\emptyset$\} e \textit{y} = \{$\emptyset$\}, \textit{z} = \{\{$\emptyset$\}, \{$\emptyset$\}\}. A utilização desses axiomas permite criar uma infinidade de conjuntos: \textit{z} = \{\{\{$\emptyset$\}, \{$\emptyset$\}\}, $\emptyset$\}, \textit{z} = \{\{\{$\emptyset$\}, \{$\emptyset$\}\}, \{$\emptyset$\}\}, etc.

