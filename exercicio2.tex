\section{Exercício 2.1}
\paragraph{Usar o axioma do Par para definir outro exemplo de conjunto.}

\begin{quote}
	Axioma Par: $\forall$\textit{x}$\forall$\textit{y}$\exists$\textit{z}$\forall$\textit{t}(\textit{t} $\in$ \textit{z} $\Leftrightarrow$ (\textit{t} = \textit{x} $\lor$ \textit{t} = \textit{y}))
	\newline
	Usando o emprego de definições explícitas abreviativas na linguagem \textfrak{S}, temos:
	\newline
	\textit{z} = \{\textit{x},\textit{y}\} \vdots\;$\forall$\textit{t}(\textit{t} $\in$ \textit{z} $\Leftrightarrow$ (\textit{t} = \textit{x} $\lor$ \textit{t} = \textit{y}))	
\end{quote}
\subsection{Exemplo de conjunto}
	Os axiomas Extensionalidade e Vazio, respectivamente $\forall$\textit{x}$\forall$\textit{y}$\forall$\textit{z}((\textit{z} $\in$ \textit{x} $\Leftrightarrow$ \textit{z} $\in$ \textinit{y}) $\Rightarrow$ \textit{x} = \textit{y})  e $\exists$\textit{x}$\forall$\textit{z}(\textit{z} $\notin$ \textit{x}) permitem a demonstração do Teorema: \textit{O conjunto x do axioma Vazio é único}.
	Utilizando esse axioma podemos escrever \textit{x} = $\emptyset$  e \textit{y} = $\emptyset$, logo \textit{z} = \{$\emptyset$\}. 
	\newline
	Se \textit{z} = \{$\emptyset$\}, podemos criar outro exemplo: \textit{x} = $\emptyset$  e \textit{y} = \{$\emptyset$\}, então teremos \textit{z} = \{$\emptyset$\ , \{$\emptyset$\}\}. Pelo axioma da Extensionalidade \{$\emptyset$\ , \{$\emptyset$\}\} = \{\{$\emptyset$\} , $\emptyset$\}.
	\newline
	Outros exemplos usando essa lógica \textit{x} = \{$\emptyset$\}  e \textit{y} = \{$\emptyset$\}, nesse caso teremos  \textit{z} = \{\{$\emptyset$\} , \{$\emptyset$\}\};
	\newline
	\textit{x} = \{\{$\emptyset$\} , \{$\emptyset$\}\}  e \textit{y} = $\emptyset$, nesse caso teremos  \textit{z} = \{\{\{$\emptyset$\} , \{$\emptyset$\}\}\ , $\emptyset$\}.