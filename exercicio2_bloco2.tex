\doublespacing
Provar que sen($\beta$ - $\gamma$) = sen($\beta$) cos($\gamma$) - cos($\beta$) sen($\gamma$).

\subsection{Provar que sen($\beta$ - $\gamma$) = sen($\beta$) cos($\gamma$) - cos($\beta$) sen($\gamma$).}
%Teorema 34 .\textit{ O problema de contorno y' = y e y(0) = $\alpha$ tem solução única, se $\alpha$ $\geq$ 0}.
De acordo com o Teorema de Euler, temos: 
\newline
exp(\textit{i$\beta$ - i$\gamma$}) = exp(\textit{i($\beta$ - $\gamma$)}) = cos($\beta$ - $\gamma$) + \textit{i}sen($\beta$ - $\gamma$).
\newline
No entanto,
\newline
exp(\textit{i$\beta$}) exp(\textit{-i$\gamma$}) = (cos($\beta$) + \textit{i}sen($\beta$)) (cos($\gamma$) - \textit{i}sen($\gamma$)) = \newline
= cos($\beta$)cos($\gamma$) - cos($\beta$)isen($\gamma$) + isen($\beta$)cos($\gamma$) - isen($\beta$)isen($\gamma$) =\newline
= cos($\beta$)cos($\gamma$) - cos($\beta$)isen($\gamma$) + isen($\beta$)cos($\gamma$) + sen($\beta$)sen($\gamma$)= \newline
= cos($\beta$)cos($\gamma$) + sen($\beta$)sen($\gamma$) + i(sen($\beta$)cos($\gamma$) - cos($\beta$)isen($\gamma$))
\newline
Usando o teorema: Sejam $\alpha$ e $\beta$ números reais quaisquer. Logo, \textit{exp($\alpha$)exp($\beta$) = exp($\alpha$ + $\beta$).}
\newline
Uma vez que \textit{exp(\textit{i$\beta$ - i$\gamma$}) = exp(\textit{i$\beta$}) exp(\textit{-i$\gamma$})}, temos
\newline
cos($\beta$ - $\gamma$) + \textit{i}sen($\beta$ - $\gamma$) = cos($\beta$)cos($\gamma$) + sen($\beta$)sen($\gamma$) + i(sen($\beta$)cos($\gamma$) - cos($\beta$)isen($\gamma$))
\newline
comparando as partes reais e imaginárias, temos as duas igualdades a seguir:\newline
cos($\beta$ - $\gamma$) = cos($\beta$)cos($\gamma$) + sen($\beta$)sen($\gamma$) e
sen($\beta$ - $\gamma$) = sen($\beta$)cos($\gamma$) - cos($\beta$)isen($\gamma$))

\subsection{Provar que sen($\beta$ - $\gamma$) = sen($\beta$) cos($\gamma$) - cos($\beta$) sen($\gamma$), usando o teorema sen($\beta$ + $\gamma$) = sen($\beta$)cos($\gamma$) + cos($\beta$)isen($\gamma$)}
Podemos escrever
sen($\beta$ - $\gamma$) = sen($\beta$ + (-$\gamma$)) = sen($\beta$)cos(-$\gamma$) + cos($\beta$)isen(-$\gamma$).
\newline
Sabemos que cos(\textit{x}) = 1 - $\frac{x^2}{2!}$ + $\frac{x^4}{4!}$ - $\frac{x^6}{6!}$ + $\frac{x^8}{8!}$ - ..., logo 
cos(\textit{-x}) = 1 - $\frac{(-x)^2}{2!}$ + $\frac{(-x)^4}{4!}$ - $\frac{(-x)^6}{6!}$ + $\frac{(-x)^8}{8!}$ - ...\newline
cos(\textit{-x}) = 1 - $\frac{x^2}{2!}$ + $\frac{x^4}{4!}$ - $\frac{x^6}{6!}$ + $\frac{x^8}{8!}$ - ..., portanto cos(\textit{x}) = cos(\textit{-x})
\newline
De forma análoga,  sen(\textit{x}) = \textit{x} - $\frac{x^3}{3!}$ + $\frac{x^5}{5!}$ - $\frac{x^7}{7!}$ + $\frac{x^9}{9!}$ - ..., logo \newline
sen(\textit{-x}) = \textit{(-x)} - $\frac{(-x)^3}{3!}$ + $\frac{(-x)^5}{5!}$ - $\frac{(-x)^7}{7!}$ + $\frac{(-x)^9}{9!}$ - ... = \textit{-x} + $\frac{x^3}{3!}$ - $\frac{x^5}{5!}$ + $\frac{x^7}{7!}$ - $\frac{x^9}{9!}$ + ...\newline
sen(\textit{-x}) = -(\textit{x} - $\frac{x^3}{3!}$ + $\frac{x^5}{5!}$ - $\frac{x^7}{7!}$ + $\frac{x^9}{9!}$ - ...), portanto sen(\textit{-x}) = -sen(\textit{x})
\newline
Então teremos cos(-$\gamma$) = cos($\gamma$) e sen(-$\gamma$) = -sen($\gamma$), logo \newline
sen($\beta$ - $\gamma$) = sen($\beta$)cos($\gamma$) - cos($\beta$)isen($\gamma$)).


